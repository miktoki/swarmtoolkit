% Generated by Sphinx.
\def\sphinxdocclass{article}
\documentclass[letterpaper,10pt,english]{sphinxhowto}
\usepackage[utf8]{inputenc}
\DeclareUnicodeCharacter{00A0}{\nobreakspace}
\usepackage{cmap}
\usepackage[T1]{fontenc}
\usepackage{babel}
\usepackage{times}
\usepackage[Bjarne]{fncychap}
\usepackage{longtable}
\usepackage{sphinx}
\usepackage{multirow}
\usepackage{eqparbox}
\usepackage{amsfonts}

\addto\captionsenglish{\renewcommand{\figurename}{Fig. }}
\addto\captionsenglish{\renewcommand{\tablename}{Table }}
\SetupFloatingEnvironment{literal-block}{name=Listing }



\title{swtools}
\date{May 11, 2016}
\release{1.0.0}
\author{Mikael Toresen}
\newcommand{\sphinxlogo}{}
\renewcommand{\releasename}{Release}
\setcounter{tocdepth}{1}
\makeindex

\makeatletter
\def\PYG@reset{\let\PYG@it=\relax \let\PYG@bf=\relax%
    \let\PYG@ul=\relax \let\PYG@tc=\relax%
    \let\PYG@bc=\relax \let\PYG@ff=\relax}
\def\PYG@tok#1{\csname PYG@tok@#1\endcsname}
\def\PYG@toks#1+{\ifx\relax#1\empty\else%
    \PYG@tok{#1}\expandafter\PYG@toks\fi}
\def\PYG@do#1{\PYG@bc{\PYG@tc{\PYG@ul{%
    \PYG@it{\PYG@bf{\PYG@ff{#1}}}}}}}
\def\PYG#1#2{\PYG@reset\PYG@toks#1+\relax+\PYG@do{#2}}

\expandafter\def\csname PYG@tok@gu\endcsname{\let\PYG@bf=\textbf\def\PYG@tc##1{\textcolor[rgb]{0.50,0.00,0.50}{##1}}}
\expandafter\def\csname PYG@tok@err\endcsname{\def\PYG@bc##1{\setlength{\fboxsep}{0pt}\fcolorbox[rgb]{1.00,0.00,0.00}{1,1,1}{\strut ##1}}}
\expandafter\def\csname PYG@tok@mo\endcsname{\def\PYG@tc##1{\textcolor[rgb]{0.13,0.50,0.31}{##1}}}
\expandafter\def\csname PYG@tok@gt\endcsname{\def\PYG@tc##1{\textcolor[rgb]{0.00,0.27,0.87}{##1}}}
\expandafter\def\csname PYG@tok@gp\endcsname{\let\PYG@bf=\textbf\def\PYG@tc##1{\textcolor[rgb]{0.78,0.36,0.04}{##1}}}
\expandafter\def\csname PYG@tok@vc\endcsname{\def\PYG@tc##1{\textcolor[rgb]{0.73,0.38,0.84}{##1}}}
\expandafter\def\csname PYG@tok@ge\endcsname{\let\PYG@it=\textit}
\expandafter\def\csname PYG@tok@mf\endcsname{\def\PYG@tc##1{\textcolor[rgb]{0.13,0.50,0.31}{##1}}}
\expandafter\def\csname PYG@tok@ow\endcsname{\let\PYG@bf=\textbf\def\PYG@tc##1{\textcolor[rgb]{0.00,0.44,0.13}{##1}}}
\expandafter\def\csname PYG@tok@kc\endcsname{\let\PYG@bf=\textbf\def\PYG@tc##1{\textcolor[rgb]{0.00,0.44,0.13}{##1}}}
\expandafter\def\csname PYG@tok@s\endcsname{\def\PYG@tc##1{\textcolor[rgb]{0.25,0.44,0.63}{##1}}}
\expandafter\def\csname PYG@tok@ne\endcsname{\def\PYG@tc##1{\textcolor[rgb]{0.00,0.44,0.13}{##1}}}
\expandafter\def\csname PYG@tok@nd\endcsname{\let\PYG@bf=\textbf\def\PYG@tc##1{\textcolor[rgb]{0.33,0.33,0.33}{##1}}}
\expandafter\def\csname PYG@tok@nc\endcsname{\let\PYG@bf=\textbf\def\PYG@tc##1{\textcolor[rgb]{0.05,0.52,0.71}{##1}}}
\expandafter\def\csname PYG@tok@c1\endcsname{\let\PYG@it=\textit\def\PYG@tc##1{\textcolor[rgb]{0.25,0.50,0.56}{##1}}}
\expandafter\def\csname PYG@tok@sh\endcsname{\def\PYG@tc##1{\textcolor[rgb]{0.25,0.44,0.63}{##1}}}
\expandafter\def\csname PYG@tok@si\endcsname{\let\PYG@it=\textit\def\PYG@tc##1{\textcolor[rgb]{0.44,0.63,0.82}{##1}}}
\expandafter\def\csname PYG@tok@m\endcsname{\def\PYG@tc##1{\textcolor[rgb]{0.13,0.50,0.31}{##1}}}
\expandafter\def\csname PYG@tok@se\endcsname{\let\PYG@bf=\textbf\def\PYG@tc##1{\textcolor[rgb]{0.25,0.44,0.63}{##1}}}
\expandafter\def\csname PYG@tok@mh\endcsname{\def\PYG@tc##1{\textcolor[rgb]{0.13,0.50,0.31}{##1}}}
\expandafter\def\csname PYG@tok@kp\endcsname{\def\PYG@tc##1{\textcolor[rgb]{0.00,0.44,0.13}{##1}}}
\expandafter\def\csname PYG@tok@gd\endcsname{\def\PYG@tc##1{\textcolor[rgb]{0.63,0.00,0.00}{##1}}}
\expandafter\def\csname PYG@tok@il\endcsname{\def\PYG@tc##1{\textcolor[rgb]{0.13,0.50,0.31}{##1}}}
\expandafter\def\csname PYG@tok@gi\endcsname{\def\PYG@tc##1{\textcolor[rgb]{0.00,0.63,0.00}{##1}}}
\expandafter\def\csname PYG@tok@nv\endcsname{\def\PYG@tc##1{\textcolor[rgb]{0.73,0.38,0.84}{##1}}}
\expandafter\def\csname PYG@tok@sx\endcsname{\def\PYG@tc##1{\textcolor[rgb]{0.78,0.36,0.04}{##1}}}
\expandafter\def\csname PYG@tok@kt\endcsname{\def\PYG@tc##1{\textcolor[rgb]{0.56,0.13,0.00}{##1}}}
\expandafter\def\csname PYG@tok@vi\endcsname{\def\PYG@tc##1{\textcolor[rgb]{0.73,0.38,0.84}{##1}}}
\expandafter\def\csname PYG@tok@gs\endcsname{\let\PYG@bf=\textbf}
\expandafter\def\csname PYG@tok@go\endcsname{\def\PYG@tc##1{\textcolor[rgb]{0.20,0.20,0.20}{##1}}}
\expandafter\def\csname PYG@tok@s1\endcsname{\def\PYG@tc##1{\textcolor[rgb]{0.25,0.44,0.63}{##1}}}
\expandafter\def\csname PYG@tok@cp\endcsname{\def\PYG@tc##1{\textcolor[rgb]{0.00,0.44,0.13}{##1}}}
\expandafter\def\csname PYG@tok@sb\endcsname{\def\PYG@tc##1{\textcolor[rgb]{0.25,0.44,0.63}{##1}}}
\expandafter\def\csname PYG@tok@bp\endcsname{\def\PYG@tc##1{\textcolor[rgb]{0.00,0.44,0.13}{##1}}}
\expandafter\def\csname PYG@tok@no\endcsname{\def\PYG@tc##1{\textcolor[rgb]{0.38,0.68,0.84}{##1}}}
\expandafter\def\csname PYG@tok@cpf\endcsname{\let\PYG@it=\textit\def\PYG@tc##1{\textcolor[rgb]{0.25,0.50,0.56}{##1}}}
\expandafter\def\csname PYG@tok@k\endcsname{\let\PYG@bf=\textbf\def\PYG@tc##1{\textcolor[rgb]{0.00,0.44,0.13}{##1}}}
\expandafter\def\csname PYG@tok@nf\endcsname{\def\PYG@tc##1{\textcolor[rgb]{0.02,0.16,0.49}{##1}}}
\expandafter\def\csname PYG@tok@o\endcsname{\def\PYG@tc##1{\textcolor[rgb]{0.40,0.40,0.40}{##1}}}
\expandafter\def\csname PYG@tok@vg\endcsname{\def\PYG@tc##1{\textcolor[rgb]{0.73,0.38,0.84}{##1}}}
\expandafter\def\csname PYG@tok@mb\endcsname{\def\PYG@tc##1{\textcolor[rgb]{0.13,0.50,0.31}{##1}}}
\expandafter\def\csname PYG@tok@cs\endcsname{\def\PYG@tc##1{\textcolor[rgb]{0.25,0.50,0.56}{##1}}\def\PYG@bc##1{\setlength{\fboxsep}{0pt}\colorbox[rgb]{1.00,0.94,0.94}{\strut ##1}}}
\expandafter\def\csname PYG@tok@kr\endcsname{\let\PYG@bf=\textbf\def\PYG@tc##1{\textcolor[rgb]{0.00,0.44,0.13}{##1}}}
\expandafter\def\csname PYG@tok@nn\endcsname{\let\PYG@bf=\textbf\def\PYG@tc##1{\textcolor[rgb]{0.05,0.52,0.71}{##1}}}
\expandafter\def\csname PYG@tok@s2\endcsname{\def\PYG@tc##1{\textcolor[rgb]{0.25,0.44,0.63}{##1}}}
\expandafter\def\csname PYG@tok@kd\endcsname{\let\PYG@bf=\textbf\def\PYG@tc##1{\textcolor[rgb]{0.00,0.44,0.13}{##1}}}
\expandafter\def\csname PYG@tok@na\endcsname{\def\PYG@tc##1{\textcolor[rgb]{0.25,0.44,0.63}{##1}}}
\expandafter\def\csname PYG@tok@ni\endcsname{\let\PYG@bf=\textbf\def\PYG@tc##1{\textcolor[rgb]{0.84,0.33,0.22}{##1}}}
\expandafter\def\csname PYG@tok@mi\endcsname{\def\PYG@tc##1{\textcolor[rgb]{0.13,0.50,0.31}{##1}}}
\expandafter\def\csname PYG@tok@sc\endcsname{\def\PYG@tc##1{\textcolor[rgb]{0.25,0.44,0.63}{##1}}}
\expandafter\def\csname PYG@tok@ch\endcsname{\let\PYG@it=\textit\def\PYG@tc##1{\textcolor[rgb]{0.25,0.50,0.56}{##1}}}
\expandafter\def\csname PYG@tok@gh\endcsname{\let\PYG@bf=\textbf\def\PYG@tc##1{\textcolor[rgb]{0.00,0.00,0.50}{##1}}}
\expandafter\def\csname PYG@tok@w\endcsname{\def\PYG@tc##1{\textcolor[rgb]{0.73,0.73,0.73}{##1}}}
\expandafter\def\csname PYG@tok@kn\endcsname{\let\PYG@bf=\textbf\def\PYG@tc##1{\textcolor[rgb]{0.00,0.44,0.13}{##1}}}
\expandafter\def\csname PYG@tok@nl\endcsname{\let\PYG@bf=\textbf\def\PYG@tc##1{\textcolor[rgb]{0.00,0.13,0.44}{##1}}}
\expandafter\def\csname PYG@tok@nb\endcsname{\def\PYG@tc##1{\textcolor[rgb]{0.00,0.44,0.13}{##1}}}
\expandafter\def\csname PYG@tok@nt\endcsname{\let\PYG@bf=\textbf\def\PYG@tc##1{\textcolor[rgb]{0.02,0.16,0.45}{##1}}}
\expandafter\def\csname PYG@tok@sd\endcsname{\let\PYG@it=\textit\def\PYG@tc##1{\textcolor[rgb]{0.25,0.44,0.63}{##1}}}
\expandafter\def\csname PYG@tok@gr\endcsname{\def\PYG@tc##1{\textcolor[rgb]{1.00,0.00,0.00}{##1}}}
\expandafter\def\csname PYG@tok@sr\endcsname{\def\PYG@tc##1{\textcolor[rgb]{0.14,0.33,0.53}{##1}}}
\expandafter\def\csname PYG@tok@cm\endcsname{\let\PYG@it=\textit\def\PYG@tc##1{\textcolor[rgb]{0.25,0.50,0.56}{##1}}}
\expandafter\def\csname PYG@tok@ss\endcsname{\def\PYG@tc##1{\textcolor[rgb]{0.32,0.47,0.09}{##1}}}
\expandafter\def\csname PYG@tok@c\endcsname{\let\PYG@it=\textit\def\PYG@tc##1{\textcolor[rgb]{0.25,0.50,0.56}{##1}}}

\def\PYGZbs{\char`\\}
\def\PYGZus{\char`\_}
\def\PYGZob{\char`\{}
\def\PYGZcb{\char`\}}
\def\PYGZca{\char`\^}
\def\PYGZam{\char`\&}
\def\PYGZlt{\char`\<}
\def\PYGZgt{\char`\>}
\def\PYGZsh{\char`\#}
\def\PYGZpc{\char`\%}
\def\PYGZdl{\char`\$}
\def\PYGZhy{\char`\-}
\def\PYGZsq{\char`\'}
\def\PYGZdq{\char`\"}
\def\PYGZti{\char`\~}
% for compatibility with earlier versions
\def\PYGZat{@}
\def\PYGZlb{[}
\def\PYGZrb{]}
\makeatother

\renewcommand\PYGZsq{\textquotesingle}

\begin{document}

\maketitle
\tableofcontents
\phantomsection\label{index::doc}

\phantomsection\label{index:index}\phantomsection\label{swtools_doc:swtools-doc}\phantomsection\label{swtools_doc:module-swtools}\phantomsection\label{swtools_doc:swtools-doc}\index{swtools (module)}

\section{Documentation}
\label{swtools_doc:swtools-index}\label{swtools_doc:documentation}\label{swtools_doc::doc}
\emph{swtools} is a toolbox intended to provide simple and quick access to
the Swarm L1 and L2 products for Python3.
\begin{description}
\item[{Main features:}] \leavevmode\begin{itemize}
\item {} 
reading and introspection of CDF (Common Data Format) files and

\end{itemize}

the containing parameters to \emph{numpy.ndarray}`s, whether the CDF's
are stored locally, within zip-files or on an ftp-server.
- shift parameters with respect to each other in time, both using a
user-defined time shift and by finding a best fit within a range
using a minimizer. Parameters can also be aligned through
interpolation to be evaluated at the same time values.
- Compute the magnetic field and its derivative in the NEC frame
from SHC ascii files.
- Convenience functions to visualize plots using \emph{matplotlib} and
\emph{mpl\_toolkits.basemap}.

\end{description}

The demo notebook, found under the demo directory gives some instructional
examples on how this package may be used, and is a good way to quickly get
started. The documentation provides a more comprehensive look at the features.
\begin{description}
\item[{Documentation shown below may also be found for each function:}] \leavevmode\begin{itemize}
\item {} 
by using the \code{help} function:

\begin{Verbatim}[commandchars=\\\{\}]
\PYG{g+gp}{\PYGZgt{}\PYGZgt{}\PYGZgt{} }\PYG{n}{help}\PYG{p}{(}\PYG{n}{swtools}\PYG{o}{.}\PYG{n}{getCDFparams}\PYG{p}{)}
\end{Verbatim}

\item {} 
in \emph{ipython} shell by typing \code{?} after the object:

\begin{Verbatim}[commandchars=\\\{\}]
\PYGZgt{}\PYGZgt{}\PYGZgt{} swtools.getCDFparams?
\end{Verbatim}

\end{itemize}

\end{description}
\index{debug\_info() (in module swtools)}

\begin{fulllineitems}
\phantomsection\label{swtools_doc:swtools.debug_info}\pysiglinewithargsret{\code{swtools.}\bfcode{debug\_info}}{\emph{activate=1}}{}
Set the verbosity level of the logger.
\begin{quote}\begin{description}
\item[{Parameters}] \leavevmode
\textbf{\texttt{activate}} (\href{https://docs.python.org/library/functions.html\#int}{\emph{\texttt{int}}}) -- Possible values:
- \code{activate\textgreater{}0} : logger set to DEBUG (default)
- \code{activate\textless{}0} : logger set to CRITICAL (virtually no logging)
- \code{activate=0} : logger set to INFO (nominal value)

\end{description}\end{quote}

\end{fulllineitems}

\index{concatenate\_values() (in module swtools)}

\begin{fulllineitems}
\phantomsection\label{swtools_doc:swtools.concatenate_values}\pysiglinewithargsret{\code{swtools.}\bfcode{concatenate\_values}}{\emph{*an}, \emph{axis=None}}{}~\phantomsection\label{swtools_doc:concatenate-values}
Concatenate nD array's along \emph{axis}.

Concatenate along last axis if none specified.

\end{fulllineitems}

\index{dl\_ftp() (in module swtools)}

\begin{fulllineitems}
\phantomsection\label{swtools_doc:swtools.dl_ftp}\pysiglinewithargsret{\code{swtools.}\bfcode{dl\_ftp}}{\emph{url='swarm-diss.eo.esa.int'}, \emph{**kwargs}}{}~\phantomsection\label{swtools_doc:dl-ftp}
Download files from ftp server.

Download files from a specified ftp url using filtering and/or
interactive specification, if not already downloaded to given location.
\begin{quote}\begin{description}
\item[{Parameters}] \leavevmode
\textbf{\texttt{url}} (\href{https://docs.python.org/library/functions.html\#str}{\emph{\texttt{str}}}) -- url of filename(s) or filename directory used to search for cdf file.

\item[{Keyword Arguments}] \leavevmode\begin{itemize}
\item {} 
\textbf{user} (\emph{str}) --
ftp server username

\item {} 
\textbf{pw} (\emph{str}) --
ftp server password

\item {} 
\textbf{dlloc} (\emph{str}) --
location to download to

\item {} 
\textbf{cdfsuffix} --
see {\hyperref[swtools_doc:getcdfparamlist]{\emph{getCDFparamlist}}}

\item {} 
\textbf{use\_filtering} --
see {\hyperref[swtools_doc:getcdflist]{\emph{getCDFlist}}}

\item {} 
\textbf{param0} --
see {\hyperref[swtools_doc:getcdfparams]{\emph{getCDFparams}}}

\item {} 
\textbf{sat} --
see {\hyperref[swtools_doc:getcdflist]{\emph{getCDFlist}}}

\item {} 
\textbf{period} --
see {\hyperref[swtools_doc:getcdflist]{\emph{getCDFlist}}}

\item {} 
\textbf{use\_passive\_mode} (\emph{bool, optional}) --
Try to set this to \code{False} if having difficulties with
connecting to ftp server (default \code{True}).

\item {} 
\textbf{use\_current} (\emph{bool, optional}) --
ignore directories named `Previous' (default \code{True}).

\item {} 
\textbf{use\_color} (\emph{bool, optional}) --
allow coloring in listing of files/directories
(default \code{True}).

\end{itemize}

\item[{Returns}] \leavevmode
Paths to files specified for download. If only one file, a string
will be returned.

\item[{Return type}] \leavevmode
str or list of str

\item[{Raises}] \leavevmode
\code{OSError} --
if unable to connect to ftp server

\end{description}\end{quote}

\end{fulllineitems}

\index{extract\_parameter() (in module swtools)}

\begin{fulllineitems}
\phantomsection\label{swtools_doc:swtools.extract_parameter}\pysiglinewithargsret{\code{swtools.}\bfcode{extract\_parameter}}{\emph{cdflist}, \emph{parameter}, \emph{**kwargs}}{}~\phantomsection\label{swtools_doc:extract-parameter}
Extract given parameter from cdf file.

Extract a parameter's values, unit and name from a list of cdf
files and concatenate values along a given axis to a
\emph{numpy.ndarray}.
\begin{quote}\begin{description}
\item[{Parameters}] \leavevmode\begin{itemize}
\item {} 
\textbf{\texttt{cdflist}} (\emph{\texttt{str or list of str}}) -- Path(s) of cdf file(s).

\item {} 
\textbf{\texttt{parameter}} (\href{https://docs.python.org/library/functions.html\#str}{\emph{\texttt{str}}}) -- Name of parameter within cdf file. If name is not known,
parameters within cdf files may be shown using
{\hyperref[swtools_doc:getcdfparamlist]{\emph{getCDFparamlist}}}.

\end{itemize}

\item[{Keyword Arguments}] \leavevmode\begin{itemize}
\item {} 
\textbf{cat} (\emph{bool}) --
concatenate parameter values (default \code{True})

\item {} 
\textbf{axis} (\emph{int}) --
concatenate along specified axis. If \code{axis=None} it will
concatenate along last axis. Ie. in a 5x4x2 array it will set
axis to 3 (default \code{None}).

\end{itemize}

\item[{Returns}] \leavevmode
list of parameter lists, which includes values, units and names
of parameters. If only one parameter, it will not be contained
within a list.

\item[{Return type}] \leavevmode
\href{https://docs.python.org/library/functions.html\#list}{list}

\item[{Raises}] \leavevmode
\code{CDFError} --
if unable to decode any cdf file specified

\end{description}\end{quote}

\end{fulllineitems}

\index{getCDFlist() (in module swtools)}

\begin{fulllineitems}
\phantomsection\label{swtools_doc:swtools.getCDFlist}\pysiglinewithargsret{\code{swtools.}\bfcode{getCDFlist}}{\emph{inloc=None}, \emph{outloc='.'}, \emph{sort\_by\_t=False}, \emph{**kwargs}}{}~\phantomsection\label{swtools_doc:getcdflist}
Get a list of cdf's from a given location if zip or cdf.
\begin{quote}\begin{description}
\item[{Parameters}] \leavevmode\begin{itemize}
\item {} 
\textbf{\texttt{inloc}} (\emph{\texttt{str, optional}}) -- Path or url of filename(s) or filename directory used to search
for cdf file (default \code{None}).

\item {} 
\textbf{\texttt{outloc}} (\emph{\texttt{str, optional}}) -- Directory path of output files if input includes zip files or
remote files. If directory does not exist it will be created
(default \code{os.curdir}).

\item {} 
\textbf{\texttt{sort\_by\_t}} (\emph{\texttt{bool, optional}}) -- Sort cdf filelist by start time of product (requires filenames
to follow standard ESA Swarm naming convention)
(default \code{False}).

\end{itemize}

\item[{Keyword Arguments}] \leavevmode\begin{itemize}
\item {} 
\textbf{cdfsuffix} --
see {\hyperref[swtools_doc:getcdfparamlist]{\emph{getCDFparamlist}}}.

\item {} 
\textbf{temp} (\emph{bool}) --
Specify whether to store cdf files extracted from zip
temporarily or not (default \code{False}).

\item {} 
\textbf{includezip} (\emph{bool}) --
Traverse zip files if cdf files already in directory. Otherwise
will only open zip files if no cdf files present.
(default \code{False}).

\item {} 
\textbf{use\_ftp} (\emph{bool}) --
specify use of ftp server to locate cdf files. Will be
overruled (implicitly set to True) if \emph{user} is specified
(see {\hyperref[swtools_doc:dl\string-ftp]{\emph{dl\_ftp}}})(default \code{False}).

\item {} 
\textbf{filter\_param} (\emph{bool}) --
Specify whether \emph{param0} (if specified, see {\hyperref[swtools_doc:getcdfparams]{\emph{getCDFparams}}}) will
be used for filtering files based on filenames or not
(default \code{False}).

\item {} 
\textbf{sat} (\emph{str or list of str}) --
Filter files based on satelite in filename, eg.
\code{sat={[}'A','C'{]}} will only include files from Alpha or
Charlie, while \code{sat='AC'} will only accept the joint
Alpha-Charlie product files (default \code{None}).

\item {} 
\textbf{period} (list of two \emph{datetime.datetime} objects) --
If a product is not (partially) within
time window specified, product will be filtered away based on
filename. Can alternatively be set using two of \emph{start\_t},
\emph{end\_t} and \emph{duration} (default \code{None}).

\item {} 
\textbf{start\_t, end\_t} (\emph{str, scalar or datetime.datetime}) --
Start/end time of filter. Scalars
will be interpreted as fractional days since MJD2000 and strings
can follow the formats \code{'yyyy-mm-dd'}, \code{'yyyymmdd'},
\code{'yyyymmddThhmmss'} or \code{'yyyymmddhhmmss'} (default \code{None}).

\item {} 
\textbf{duration} (\emph{scalar}) --
Duration after \emph{start\_t} or before \emph{end\_t} in fractional days
(default \code{None}).

\item {} 
\textbf{param0} --
see {\hyperref[swtools_doc:getcdfparams]{\emph{getCDFparams}}} .

\end{itemize}

\item[{Returns}] \leavevmode
\textbf{cdfl} --
list of strings of absolute paths to cdf files found in \emph{inloc}.

\item[{Return type}] \leavevmode
\href{https://docs.python.org/library/functions.html\#list}{list}

\end{description}\end{quote}
\paragraph{Notes}

The \emph{inloc} argument will be overwritten by any \emph{inloc} entry in
\emph{kwargs}.

If \emph{start\_t}. \emph{end\_t} and \emph{duration} are all specified, \emph{duration}
will be ignored.

\end{fulllineitems}

\index{getCDFparamlist() (in module swtools)}

\begin{fulllineitems}
\phantomsection\label{swtools_doc:swtools.getCDFparamlist}\pysiglinewithargsret{\code{swtools.}\bfcode{getCDFparamlist}}{\emph{cdflist, cdfsuffix={[}'DBL', `CDF'{]}}}{}~\phantomsection\label{swtools_doc:getcdfparamlist}
List parameters in the given files if unique.

Prints a list of parameters for each unique product type based on
filename, and returns a list with corresponding information.
\begin{quote}\begin{description}
\item[{Parameters}] \leavevmode\begin{itemize}
\item {} 
\textbf{\texttt{cdflist}} (\emph{\texttt{str or list of str}}) -- Path(s) of cdf file(s).

\item {} 
\textbf{\texttt{cdfsuffix}} (\emph{\texttt{str or list of str, optional}}) -- case-insensitive list of strings of file extensions to accept
as valid cdf files. If none found, zip files will be explored
for the same file suffixes (default \code{{[}'DBL','CDF'{]}}).

\end{itemize}

\item[{Raises}] \leavevmode
\code{CDFError} --
if unable to decode any cdf file specified.

\end{description}\end{quote}

\end{fulllineitems}

\index{getCDFparams() (in module swtools)}

\begin{fulllineitems}
\phantomsection\label{swtools_doc:swtools.getCDFparams}\pysiglinewithargsret{\code{swtools.}\bfcode{getCDFparams}}{\emph{inloc}, \emph{*params}, \emph{**kwargs}}{}~\phantomsection\label{swtools_doc:getcdfparams}
Extract parameters from cdf input as a list.

See {\hyperref[swtools_doc:getcdflist]{\emph{getCDFlist}}} for a list of keyword arguments not specified here.
\begin{quote}\begin{description}
\item[{Parameters}] \leavevmode\begin{itemize}
\item {} 
\textbf{\texttt{inloc}} (\href{https://docs.python.org/library/functions.html\#str}{\emph{\texttt{str}}}) -- Path or url of filename(s) or filename directory used to search for
cdf file.

\item {} 
\textbf{\texttt{params}} (\href{https://docs.python.org/library/functions.html\#str}{\emph{\texttt{str}}}) -- Names of parameters to be extracted from \emph{inloc}. If names are not
known, parameters within cdf files may be shown using
{\hyperref[swtools_doc:getcdfparamlist]{\emph{getCDFparamlist}}}. Multiple parameters should be given as
separate, comma-separated strings.

\end{itemize}

\item[{Keyword Arguments}] \leavevmode
\textbf{param0} (\emph{str}) --
specify parameter name to use for filtering of files. If none
specified first parameter in \emph{params} is assumed. Will not be
used unless \code{filter\_param=True} is specified.

\item[{Returns}] \leavevmode
{\hyperref[swtools_doc:parameter]{\emph{Parameter}}}`s as ordered in \emph{params}. If only one parameter is
specified, {\hyperref[swtools_doc:parameter]{\emph{Parameter}}} will not be contained within a list.

\item[{Return type}] \leavevmode
\href{https://docs.python.org/library/functions.html\#list}{list}

\item[{Raises}] \leavevmode
\code{ValueError} --
if no parameters specified

\end{description}\end{quote}
\paragraph{Notes}

Other keyword arguments are passed on to {\hyperref[swtools_doc:getcdflist]{\emph{getCDFlist}}},
{\hyperref[swtools_doc:extract\string-parameter]{\emph{extract\_parameter}}} and {\hyperref[swtools_doc:dl\string-ftp]{\emph{dl\_ftp}}} where applicable.


\strong{See also:}


{\hyperref[swtools_doc:swtools.getCDFlist]{\emph{\code{getCDFlist()}}}}, {\hyperref[swtools_doc:swtools.extract_parameter]{\emph{\code{extract\_parameter()}}}}, {\hyperref[swtools_doc:swtools.dl_ftp]{\emph{\code{dl\_ftp()}}}}



\end{fulllineitems}

\index{param\_peek() (in module swtools)}

\begin{fulllineitems}
\phantomsection\label{swtools_doc:swtools.param_peek}\pysiglinewithargsret{\code{swtools.}\bfcode{param\_peek}}{\emph{in\_arr\_cdfl}, \emph{parameter=None}, \emph{n\_show=5}, \emph{axis=0}, \emph{cataxis=None}}{}~\phantomsection\label{swtools_doc:param-peek}
Show some values contained in given cdf(s)/array.

Print values from a cdf file, {\hyperref[swtools_doc:parameter]{\emph{Parameter}}}, list of cdf files or
\emph{numpy.ndarray}:
\begin{itemize}
\item {} 
cdf input or {\hyperref[swtools_doc:parameter]{\emph{Parameter}}} only: parameter name and units

\item {} 
1D-array only :  first and last \emph{n\_show} values

\item {} 
all : shape, max, min, mean and median values.
number of zeros, NaN's, and largest jumps along given axis

\end{itemize}
\begin{quote}\begin{description}
\item[{Parameters}] \leavevmode\begin{itemize}
\item {} 
\textbf{\texttt{in\_arr\_cdfl}} (str or list of str or \emph{numpy.ndarray} or {\hyperref[swtools_doc:parameter]{\emph{Parameter}}}) -- path to cdf file(s), or array.

\item {} 
\textbf{\texttt{parameter}} (\emph{\texttt{str, optional}}) -- if input is cdf file(s), parameter name should be specified. If
names are not known, parameters within cdf files may be shown
using {\hyperref[swtools_doc:getcdfparamlist]{\emph{getCDFparamlist}}}. (default \code{None}).

\item {} 
\textbf{\texttt{n\_show}} (\emph{\texttt{int, optional}}) -- number of values to show from start and end of 1D-array
(default \code{5}).

\item {} 
\textbf{\texttt{axis}} (\href{https://docs.python.org/library/functions.html\#int}{\emph{\texttt{int}}}) -- axis to view largest jumps over (default \code{0}).

\item {} 
\textbf{\texttt{cataxis}} (\href{https://docs.python.org/library/functions.html\#int}{\emph{\texttt{int}}}) -- axis to concatenate over. If \code{cataxis=None} it will concatenate
along last axis, e.g. in a 5x4x2 array it will set axis to 3
(default \code{None}).

\end{itemize}

\end{description}\end{quote}

\end{fulllineitems}

\index{read\_sp3() (in module swtools)}

\begin{fulllineitems}
\phantomsection\label{swtools_doc:swtools.read_sp3}\pysiglinewithargsret{\code{swtools.}\bfcode{read\_sp3}}{\emph{fname}, \emph{doctype=2}, \emph{SI\_units=True}}{}~\phantomsection\label{swtools_doc:read-sp3}
Read SP3 ascii files to array.

Read orbital format `Standard Product \# 3' (SP3) to numpy array of
two SP3 document types  as shown by example 1 and example 2 in
\href{https://igscb.jpl.nasa.gov/igscb/data/format/sp3\_docu.txt}{https://igscb.jpl.nasa.gov/igscb/data/format/sp3\_docu.txt} . Output
may be converted to SI units.
\begin{quote}\begin{description}
\item[{Parameters}] \leavevmode\begin{itemize}
\item {} 
\textbf{\texttt{fname}} (\href{https://docs.python.org/library/functions.html\#str}{\emph{\texttt{str}}}) -- Path of SP3 file.

\item {} 
\textbf{\texttt{doctype}} (\href{https://docs.python.org/library/functions.html\#int}{\emph{\texttt{int}}}) -- 
Two SP3 document formats:
\begin{enumerate}
\item {} 
only position at given timestamps

\item {} 
position and velocity at given timestamps with
rate-of-change of clock correction

\end{enumerate}

\emph{doctype} corresponds to these two options (default \code{2}).


\item {} 
\textbf{\texttt{SI\_units}} (\href{https://docs.python.org/library/functions.html\#bool}{\emph{\texttt{bool}}}) -- Convert to SI-units from SP3 units (default \code{True}).

\end{itemize}

\item[{Returns}] \leavevmode
\code{{[}x, y, z, t, header{]}} for \code{doctype=1},
\code{{[}x, y, z, vx, vy, vz, dt, t, header{]}} for \code{doctype=2},
where \code{header} is the first 22 lines of the SP3 document as a
string.

\item[{Return type}] \leavevmode
list of \emph{numpy.ndarray}

\item[{Raises}] \leavevmode
\code{EOFError} --
If the specified SP3 file is empty.

\end{description}\end{quote}

\end{fulllineitems}

\index{read\_EFI\_prov\_txt() (in module swtools)}

\begin{fulllineitems}
\phantomsection\label{swtools_doc:swtools.read_EFI_prov_txt}\pysiglinewithargsret{\code{swtools.}\bfcode{read\_EFI\_prov\_txt}}{\emph{fname}, \emph{*params}, \emph{filter\_nominal=False}}{}~\phantomsection\label{swtools_doc:read-efi-prov-txt}
Read provisional EFI products.

Read an ascii file containing provisional EFI products.
\begin{quote}\begin{description}
\item[{Parameters}] \leavevmode\begin{itemize}
\item {} 
\textbf{\texttt{fname}} (\href{https://docs.python.org/library/functions.html\#str}{\emph{\texttt{str}}}) -- Path and filename of provisional EFI ascii file.

\item {} 
\textbf{\texttt{params}} -- 
Names of parameters to be extracted from \emph{fname}. Possible values:
- `timestamp'
- `latitude'
- `longitude'
- `radius'
- `n'
- `t\_elec'
- `u\_sc'
- `flag'

If no parameters are specified, all will be returned in a
dictionary. Multiple parameters should be given as separate,
comma-separated strings.


\item {} 
\textbf{\texttt{filter\_nominal}} (\href{https://docs.python.org/library/functions.html\#bool}{\emph{\texttt{bool}}}) -- Only extract data where \code{Flag=1} (default \code{False}).

\end{itemize}

\item[{Returns}] \leavevmode
If parameters specified, returns list of \emph{numpy.ndarray}`s; otherwise
it will return dictionary of all parameter \emph{numpy.ndarray}`s.

\item[{Return type}] \leavevmode
List or dictionary

\end{description}\end{quote}
\paragraph{Notes}

This function has not been subject to any optimization, and is slow.
Multiple file support or data concatenation is as of yet not
implemented.

\end{fulllineitems}

\index{unzip\_file() (in module swtools)}

\begin{fulllineitems}
\phantomsection\label{swtools_doc:swtools.unzip_file}\pysiglinewithargsret{\code{swtools.}\bfcode{unzip\_file}}{\emph{input\_file}, \emph{output\_loc}}{}~\phantomsection\label{swtools_doc:unzip-file}
Unzip file to location if files in zip are not already present.
\begin{quote}\begin{description}
\item[{Parameters}] \leavevmode\begin{itemize}
\item {} 
\textbf{\texttt{input\_file}} (\href{https://docs.python.org/library/functions.html\#str}{\emph{\texttt{str}}}) -- path of zip file to extract content from.

\item {} 
\textbf{\texttt{output\_loc}} (\href{https://docs.python.org/library/functions.html\#str}{\emph{\texttt{str}}}) -- output path to extract content to.

\end{itemize}

\item[{Returns}] \leavevmode
\textbf{z}

\item[{Return type}] \leavevmode
\code{zipfile.ZipFile} object

\end{description}\end{quote}

\end{fulllineitems}

\index{Parameter (class in swtools)}

\begin{fulllineitems}
\phantomsection\label{swtools_doc:swtools.Parameter}\pysiglinewithargsret{\strong{class }\code{swtools.}\bfcode{Parameter}}{\emph{values}, \emph{unit='`}, \emph{name='`}}{}~\phantomsection\label{swtools_doc:parameter}
Container for a single parameter.

Has 3 attributes:
\begin{itemize}
\item {} 
values

\item {} 
name

\item {} 
unit

\end{itemize}

\emph{values} can also be accessed by calling the parameter
(eg. \code{myparam()}) or by accessing its indices (eg. \code{myparam{[}2{]}})

\end{fulllineitems}

\index{align\_param() (in module swtools)}

\begin{fulllineitems}
\phantomsection\label{swtools_doc:swtools.align_param}\pysiglinewithargsret{\code{swtools.}\bfcode{align\_param}}{\emph{p1}, \emph{p2}, \emph{t1}, \emph{t2}, \emph{k=3}, \emph{align\_to=False}}{}~\phantomsection\label{swtools_doc:align-param}\begin{quote}

Interpolate parameters such that time values overlap.
\end{quote}

Given parameter arrays \(p_1(t_1)\) and \(p_2(t_2)\) with
their corresponding time arrays one can downsample the most
frequently sampled array, and align the arrays wrt. time (using
interpolation) such that only one time array \(t\) is required
for \(p_1\) and \(p_2\). Only overlapping temporal regions
will be utilized. Upsampling may be forced using the \emph{align\_to}
argument.
\begin{quote}\begin{description}
\item[{Parameters}] \leavevmode\begin{itemize}
\item {} 
\textbf{\texttt{p1}} (\emph{\texttt{1D numpy.ndarray}}) -- First parameter to align.

\item {} 
\textbf{\texttt{p2}} (\emph{\texttt{1D numpy.ndarray}}) -- Second parameter to align.

\item {} 
\textbf{\texttt{t1}} (\emph{\texttt{1D numpy.ndarray of datetime.datetime objects}}) -- time values of \emph{p1}. Should have same length as \emph{p1}. If
\code{align\_to=True}, \emph{t1} will set the output frequency.

\item {} 
\textbf{\texttt{t2}} (\emph{\texttt{1D numpy.ndarray of datetime.datetime objects}}) -- time values of \emph{p2}. Should have same length as \emph{p2}.

\item {} 
\textbf{\texttt{k}} (\emph{\texttt{int, optional}}) -- Degree of the smoothing spline. Must be \code{1 \textless{}= k \textless{}= 5}
(default \code{3}).

\item {} 
\textbf{\texttt{align\_to}} (\emph{\texttt{bool, optional}}) -- Align \emph{p2} to \emph{p1} independent of respective frequency, instead of
downsampling to lowest frequency of the two. This may be useful for
aligning multiple parameters to a specific frequency, for
upsampling, or for handling datasets where one of the parameters is
not uniformly sampled (default \code{False}).

\end{itemize}

\item[{Returns}] \leavevmode
(p1',p2',t) where \emph{p1'} and \emph{p2'} are sampled at \emph{t} instead of at
\emph{t1} or \emph{t2}. p1',p2' and t are \emph{numpy.ndarray's}. As only the
temporal overlap is utilized, the temporal span of the array will
in general be less than or equal to the smallest temporal span of
the two.

\item[{Return type}] \leavevmode
\href{https://docs.python.org/library/functions.html\#tuple}{tuple}

\item[{Raises}] \leavevmode
\code{ValueError}
\begin{itemize}
\item {} 
if respective p,t pairs are not of same length.

\item {} 
if length of t array is less than spline order.

\item {} 
if time arrays are not ordered ascending.

\item {} 
if unable to interpolate arrays.

\end{itemize}

\end{description}\end{quote}
\paragraph{Notes}

To be able to interpolate, the spline order must be less than the
total number of time steps. This funtion assumes uniform sampling,
and the sampling times are currently solely determined by the step
length between the first two timesteps. If one of the parameters is
uniformly sampled while the other is not, or one of the arrays
contains non-finite numbers, \code{align\_to=True} may be used with
the finite, uniformly sampled parameter as \emph{p1}.


\strong{See also:}


{\hyperref[swtools_doc:swtools.plot_align]{\emph{\code{plot\_align()}}}}



\end{fulllineitems}

\index{shift\_param() (in module swtools)}

\begin{fulllineitems}
\phantomsection\label{swtools_doc:swtools.shift_param}\pysiglinewithargsret{\code{swtools.}\bfcode{shift\_param}}{\emph{p1}, \emph{p2}, \emph{t1}, \emph{t2}, \emph{delta\_t=0}, \emph{dt\_lim=(-20}, \emph{20)}, \emph{v=1}, \emph{spline\_points=10000000.0}, \emph{eval\_width=None}, \emph{k=3}, \emph{auto=False}, \emph{useminos=True}, \emph{imincall=10000.0}, \emph{bins=1000.0}, \emph{return\_delta=False}, \emph{show=False}, \emph{ext=2}}{}~\phantomsection\label{swtools_doc:shift-param}
Return values of \emph{p1} and \emph{p2} shifted by \emph{delta\_t}.

Shift a parameter \(p_1(t_1)\) wrt. a second parameter
\(p_2(t_2)\) by a time step \(\Delta t\). The shift can
also be done automatically to find best fit by using a minimizer
based on `SEAL Minuit'(\emph{iminuit}) with interpolated values.
\begin{quote}\begin{description}
\item[{Parameters}] \leavevmode\begin{itemize}
\item {} 
\textbf{\texttt{p1}} (\emph{\texttt{1D numpy.ndarray}}) -- Parameter to be shifted by \emph{delta\_t}

\item {} 
\textbf{\texttt{p2}} (\emph{\texttt{1D numpy.ndarray}}) -- parameter to shift \emph{p1} with respect to.

\item {} 
\textbf{\texttt{t1}} (\emph{\texttt{1D numpy.ndarray of datetime.datetime objects}}) -- time values of \emph{p1}. Should have same length as \emph{p1}.

\item {} 
\textbf{\texttt{t2}} (\emph{\texttt{1D numpy.ndarray of datetime.datetime objects}}) -- time values of \emph{p2}. Should have same length as \emph{p2}.

\item {} 
\textbf{\texttt{delta\_t}} (\emph{\texttt{int,float, optional}}) -- time shift in seconds to shift \emph{p1} by. If used together with the
argument \emph{auto=True}, this value will be used as a first guess to
the best fit. It can then be set to \emph{None} if \emph{dt\_lim} is set.
A middle value will then be assumed (default \code{0}).

\item {} 
\textbf{\texttt{dt\_lim}} (\emph{\texttt{int/float list\_like of length 2 or int/float.}}) -- Maximum and minimum value of \emph{delta\_t} in seconds allowed for
minimizing algorithm. If \emph{dt\_lim} is a number, symmetry round
\emph{delta\_t} will be assumed, eg \code{{[}delta\_t-dt\_lim,delta\_t+dt\_lim{]}}.
\emph{int} or \emph{float} must be non-negative. If \code{dt\_lim = None} it
will be set to \code{dt\_lim=(delta\_t - ((1-eval\_ratio)/2)*abs(delta\_t),
delta\_t + ((1-eval\_ratio)/2)*abs(delta\_t))}, where
\code{eval\_ratio=eval\_width/len(p1)} (default \code{(-20,20)}).

\item {} 
\textbf{\texttt{v}} (\emph{\texttt{int, optional}}) -- Verbosity level of function. \code{0 \textless{}= v \textless{}= 2} (default \code{1}).

\item {} 
\textbf{\texttt{spline\_points}} (\emph{\texttt{int, optional}}) -- Number of points used to make a spline fit of p1 with. Number
will be reduced if p1 has fewer points. Float values will be
truncated (default \code{1e7}).

\item {} 
\textbf{\texttt{eval\_width}} (\emph{\texttt{int, optional}}) -- Number of points in time to compare \emph{p1} and \emph{p2} values,
centered around the value of \code{t1+delta\_t}. Number will be
reduced by increasing span of dt\_lim to accommodate for all
possible values of \emph{delta\_t}. If set to \code{eval\_width=None} a
width corresponding to 60\% of the length of p2 will be used
(default \code{None}).

\item {} 
\textbf{\texttt{k}} (\emph{\texttt{int, optional}}) -- Degree of the smoothing spline. Must be \code{1 \textless{}= k \textless{}= 5}.

\item {} 
\textbf{\texttt{auto}} (\emph{\texttt{bool, optional}}) -- Use minimizer to find best fit for \emph{delta\_t} (default \code{False}).

\item {} 
\textbf{\texttt{useminos}} (\href{https://docs.python.org/library/functions.html\#bool}{\emph{\texttt{bool}}}) -- If \code{auto=True},run
\href{http://iminuit.readthedocs.org/en/latest/api.html\#iminuit.Minuit.minos}{minos}
(default \code{True})

\item {} 
\textbf{\texttt{imincall}} (\href{https://docs.python.org/library/functions.html\#int}{\emph{\texttt{int}}}) -- If \code{auto=True}, number of calls to migrad/minos. Float values
will be truncated (default \code{1e4})

\item {} 
\textbf{\texttt{bins}} (\href{https://docs.python.org/library/functions.html\#int}{\emph{\texttt{int}}}) -- If \emph{auto=True}, number of bins for profile of solution space (if
no solution is found from initial \emph{delta\_t}, divide dt\_lim into
\emph{bins}, and find best solution out of these). Also applicable for
when visualizing profile using \code{show=True}. Float values will
be truncated (default \code{1e3}).

\item {} 
\textbf{\texttt{return\_delta}} (\href{https://docs.python.org/library/functions.html\#bool}{\emph{\texttt{bool}}}) -- return \emph{delta\_t} as output (default \code{False}).

\item {} 
\textbf{\texttt{show}} (\href{https://docs.python.org/library/constants.html\#False}{\emph{\texttt{False}}}) -- show solution profile in a plot (see iminuit \href{http://iminuit.readthedocs.org/en/latest/api.html\#iminuit.Minuit.draw\_profile}{draw\_profile}
)(default \code{False}).

\item {} 
\textbf{\texttt{ext}} (\href{https://docs.python.org/library/functions.html\#int}{\emph{\texttt{int}}}) -- 
handling of values outside interpolation region:
\begin{itemize}
\item {} 
extrapolation = 0

\item {} 
set to zero = 1

\item {} 
raise error = 2

\item {} 
set to constant(equal to boundary) = 3

\end{itemize}


\end{itemize}

\item[{Returns}] \leavevmode
\begin{itemize}
\item {} 
a tuple of numpy.ndarray's \code{(p1,p2,t1+delta\_t,t2)} are returned.

\item {} 
\emph{Only values with temporal overlap are returned. Output will be of}

\item {} 
equal length. If \emph{return\_delta=True}, a tuple

\item {} 
\code{(p1,p2,t1+delta\_t,t2,delta\_t)} will be returned, with delta\_t as

\item {} 
\emph{float.}

\end{itemize}


\item[{Raises}] \leavevmode\begin{itemize}
\item {} 
\code{ValueError}
\begin{itemize}
\item {} 
if length's are incompatible

\item {} 
if \emph{eval\_width}\textgreater{}length of p2

\item {} 
if neither \emph{delta\_t} nor \emph{dt\_lim} are provided.

\item {} 
if \code{delta\_t=None} and \emph{dt\_lim} is a number.

\item {} 
if \emph{dt\_lim} is negative

\end{itemize}

\item {} 
\code{IndexError} --
if \emph{dt\_lim} has length less than 2.

\end{itemize}

\end{description}\end{quote}
\paragraph{Notes}

This function assumes uniform sampling rate, and may not give
desired results if this is not the case. As minimizing functions
can be non-trivial, some tweaking of arguments may be necessary
to get optimal results.


\strong{See also:}


{\hyperref[swtools_doc:swtools.align_param]{\emph{\code{align\_param()}}}}, {\hyperref[swtools_doc:swtools.where_overlap]{\emph{\code{where\_overlap()}}}}



\end{fulllineitems}

\index{where\_overlap() (in module swtools)}

\begin{fulllineitems}
\phantomsection\label{swtools_doc:swtools.where_overlap}\pysiglinewithargsret{\code{swtools.}\bfcode{where\_overlap}}{\emph{t1}, \emph{t2}, \emph{delta\_t=0}}{}
Find overlap between two datetime arrays, where one array may be
shifted by \emph{delta\_t}.

This is essentially a convenience function to access
\code{spacepy.toolbox.tOverlap(t1+delta\_t,t2,presort=True)}.

\end{fulllineitems}

\index{fourier\_transform() (in module swtools)}

\begin{fulllineitems}
\phantomsection\label{swtools_doc:swtools.fourier_transform}\pysiglinewithargsret{\code{swtools.}\bfcode{fourier\_transform}}{\emph{param}, \emph{dt\_t}, \emph{norm=None}}{}
Fourier transformation of 1d array with corresponding dt information.
\begin{quote}\begin{description}
\item[{Parameters}] \leavevmode\begin{itemize}
\item {} 
\textbf{\texttt{param}} (\emph{\texttt{array\_like}}) -- input parameter

\item {} 
\textbf{\texttt{dt\_t}} (\emph{\texttt{float, datetime.timedelta or numpy.ndarray of datetime.datetime}}) -- sample time or array of parameter sampling times.

\item {} 
\textbf{\texttt{norm}} (\emph{\texttt{\{None,'ortho'\}}}) -- None : no scaling
`ortho': direct fourier transform scaled by \code{1/sqrt(n)}, with
\code{n} being the length of \emph{param} (default \code{None}).

\end{itemize}

\item[{Returns}] \leavevmode
\emph{numpy.ndarray} of fourier transform of param, and \emph{numpy.ndarray} of
corresponding frequencies.

\item[{Return type}] \leavevmode
\href{https://docs.python.org/library/functions.html\#tuple}{tuple}

\item[{Raises}] \leavevmode
\code{TypeError} --
if \emph{dt\_t} is array and content is not \emph{datetime.datetime} objects

\end{description}\end{quote}
\paragraph{Notes}

Requires uniform temporal sampling.

\end{fulllineitems}

\index{cyclic2rising() (in module swtools)}

\begin{fulllineitems}
\phantomsection\label{swtools_doc:swtools.cyclic2rising}\pysiglinewithargsret{\code{swtools.}\bfcode{cyclic2rising}}{\emph{a, lim={[}-90, 90{]}}}{}~\begin{description}
\item[{Returns an array of monotonic rising values (requires first indices}] \leavevmode
to be rising, and assumes approx. equidistant points).

\end{description}
\begin{quote}\begin{description}
\item[{Parameters}] \leavevmode\begin{itemize}
\item {} 
\textbf{\texttt{a}} (\emph{\texttt{array\_like}}) -- array of smooth cyclic values to be made monotonic rising.

\item {} 
\textbf{\texttt{lim}} (\href{https://docs.python.org/library/functions.html\#list}{\emph{\texttt{list}}}) -- list of extremal(min,max) values within which \emph{a} is cyclic
(default \code{{[}-90,90{]}}).

\end{itemize}

\item[{Returns}] \leavevmode
array of monotonic rising values

\item[{Return type}] \leavevmode
\emph{numpy.ndarray}

\end{description}\end{quote}

\end{fulllineitems}

\index{rising2cyclic() (in module swtools)}

\begin{fulllineitems}
\phantomsection\label{swtools_doc:swtools.rising2cyclic}\pysiglinewithargsret{\code{swtools.}\bfcode{rising2cyclic}}{\emph{a, lim={[}-90, 90{]}}}{}
Returns an array of cyclic values between two extremal values
(requires first indices to be rising)
\begin{quote}\begin{description}
\item[{Parameters}] \leavevmode\begin{itemize}
\item {} 
\textbf{\texttt{a}} (\emph{\texttt{array\_like}}) -- array of monotonic rising values to be made cyclic

\item {} 
\textbf{\texttt{lim}} (\href{https://docs.python.org/library/functions.html\#list}{\emph{\texttt{list}}}) -- list of extremal(min,max) values to make array cyclic within
(default \code{{[}-90,90{]}}).

\end{itemize}

\item[{Returns}] \leavevmode
array of cyclic values

\item[{Return type}] \leavevmode
\emph{numpy.ndarray}

\end{description}\end{quote}

\end{fulllineitems}

\index{interpolate2d\_sphere() (in module swtools)}

\begin{fulllineitems}
\phantomsection\label{swtools_doc:swtools.interpolate2d_sphere}\pysiglinewithargsret{\code{swtools.}\bfcode{interpolate2d\_sphere}}{\emph{lat\_rad}, \emph{lon\_rad}, \emph{param}, \emph{**kwargs}}{}
Interpolate on sphere using radians

Convenience function to call \href{http://docs.scipy.org/doc/scipy/reference/generated/scipy.interpolate.RectSphereBivariateSpline.html}{RectSphereBivariateSpline}
\begin{quote}\begin{description}
\item[{Parameters}] \leavevmode\begin{itemize}
\item {} 
\textbf{\texttt{lat\_rad}} (\emph{\texttt{array\_like}}) -- 1-D array of latitude coordinates in strictly ascending order.
Coordinates must be given in radians, and lie within \code{(0, pi)}.

\item {} 
\textbf{\texttt{lon\_rad}} (\emph{\texttt{array\_like}}) -- 1-D array of longitude coordinates in strictly ascending order.
Coordinates must be given in radians and lie within the interval
\code{(0, 2*pi)}.

\item {} 
\textbf{\texttt{param}} (\emph{\texttt{array\_like}}) -- 2-D array of parameter with shape \code{(lat\_rad.size, lon\_rad.size)}

\end{itemize}

\item[{Returns}] \leavevmode
Spline function to be used for evaluation of interpolation

\item[{Return type}] \leavevmode
\emph{scipy.interpolate.RectSphereBivariateSpline}

\end{description}\end{quote}
\paragraph{Notes}

Keyword arguments passed on to \href{http://docs.scipy.org/doc/scipy/reference/generated/scipy.interpolate.RectSphereBivariateSpline.html}{RectSphereBivariateSpline}.

\end{fulllineitems}

\index{where\_diff() (in module swtools)}

\begin{fulllineitems}
\phantomsection\label{swtools_doc:swtools.where_diff}\pysiglinewithargsret{\code{swtools.}\bfcode{where\_diff}}{\emph{values, atol=None, rtol=None, pdiff={[}75, 25{]}, axis=0, no\_jump=False}}{}
Get indices of values which are significantly different from the
preceding values.

Function to find discontinuities using absolute tolerance, relative
tolerance and percentile differences over an array.
\begin{quote}\begin{description}
\item[{Parameters}] \leavevmode\begin{itemize}
\item {} 
\textbf{\texttt{values}} (\emph{\texttt{array\_like}}) -- input array to be evaluated

\item {} 
\textbf{\texttt{atol}} (\emph{\texttt{float, optional}}) -- absolute tolerance such that where the difference between any value
and its preceeding value is larger than \emph{atol} will be flagged as a
discontinuity. May be combined with \emph{rtol} to only flag
intersection of \emph{atol} and \emph{rtol} (default \code{None}).

\item {} 
\textbf{\texttt{rtol}} (\emph{\texttt{float, optional}}) -- relative tolerance such that where the difference between any value
and its preceeding value divided by its value is larger than \emph{rtol}
, it will be flagged as a discontinuity. May be combined with
\emph{atol} to only flag intersection of \emph{atol} and  \emph{rtol}
(default \code{None}).

\item {} 
\textbf{\texttt{pdiff}} (\emph{\texttt{list of float of length 2, optional}}) -- Two values between 0 and 100. The percentile difference such that
where the difference between any value and its preceeding value is
larger than the difference between the values of the two
percentiles of the data, it will be flagged as a discontinuity
(default \code{{[}75,25{]}}).

\item {} 
\textbf{\texttt{axis}} (\href{https://docs.python.org/library/functions.html\#int}{\emph{\texttt{int}}}) -- Axis in array over which to evaluate (default \code{0}).

\item {} 
\textbf{\texttt{no\_jump}} (\href{https://docs.python.org/library/functions.html\#bool}{\emph{\texttt{bool}}}) -- Flag continuities instead of discontinuities (default \code{False}).

\end{itemize}

\item[{Returns}] \leavevmode
Indices of flagged values

\item[{Return type}] \leavevmode
ndarray or tuple of \emph{numpy.ndarrays}

\end{description}\end{quote}

\end{fulllineitems}

\index{get\_Bnec() (in module swtools)}

\begin{fulllineitems}
\phantomsection\label{swtools_doc:swtools.get_Bnec}\pysiglinewithargsret{\code{swtools.}\bfcode{get\_Bnec}}{\emph{shc\_fn}, \emph{latitude}, \emph{longitude}, \emph{cols='all'}, \emph{lmax=-1}, \emph{lmin=-1}, \emph{lmin\_file=1}, \emph{r=1}, \emph{h=0}, \emph{t\_out={[}{]}}, \emph{k=-1}, \emph{dB=False}, \emph{ext=2}}{}
Compute magnetic field components in NEC-frame from SHC ascii file.

Get computation of the magnetic field components or its derivative
for given latitude and longitude in the North-East-Center reference
system given gaussian spherical harmonics coefficients file.
\begin{quote}\begin{description}
\item[{Parameters}] \leavevmode\begin{itemize}
\item {} 
\textbf{\texttt{shc\_fn}} (\href{https://docs.python.org/library/functions.html\#str}{\emph{\texttt{str}}}) -- Path of input SHC ascii file

\item {} 
\textbf{\texttt{latitude}} (\emph{\texttt{array\_like}}) -- latitude values to evaluate magnetic field at

\item {} 
\textbf{\texttt{longitude}} (\emph{\texttt{array\_like}}) -- longitude values to evaluate magnetic field at

\item {} 
\textbf{\texttt{cols}} (\emph{\texttt{list\_like, optional}}) -- List of columns to read from file. This should correspond to the
columns the different times values coefficients will be
read from. In a standard SHC file the first two columns (0 and 1)
correspond to the degree (\code{l}) and order (\code{m}) of the
harmonic and should not be included in \emph{cols}. As such the default
value \code{cols='all'} corresponds to \code{cols=range(2,2+N\_times)},
where \emph{N\_times} is the number of time snapshots in the file.

\item {} 
\textbf{\texttt{lmin}} (\emph{\texttt{lmax,}}) -- Maximum and minimum  degree of harmonics \(l_{max}\)
(\(l_{min}\)). If non-positive, suitable values will be set
based on the number of coefficients (default \code{-1}).

\item {} 
\textbf{\texttt{lmin\_file}} (\emph{\texttt{int, optional}}) -- Lowest value of degree in SHC file (default \code{1}).

\item {} 
\textbf{\texttt{r}} (\emph{\texttt{float, optional}}) -- Fractional radius at which to evaluate magnetic field. This is the
radius divided by the reference radius 6371.2 km
(see also \emph{h})(default \code{1}).

\item {} 
\textbf{\texttt{h}} (\emph{\texttt{float, optional}}) -- Height(in km) above reference radius 6371.2 km at which to evaluate
magnetic field. A non-zero value of \emph{h} will overwrite any
value of \emph{r} (default \code{0}).

\item {} 
\textbf{\texttt{t\_out}} (\emph{\texttt{datetime.datetime, scalar or datetime/scalar list, optional}}) -- Times at which to evaluate magnetic field. Float values should
correspond to fractional years. If left empty, times will be taken
from the SHC file (default \code{{[}{]}}).

\item {} 
\textbf{\texttt{k}} (\href{https://docs.python.org/library/functions.html\#int}{\emph{\texttt{int}}}) -- 
Spline order for temporal interpolation. If not set, spline\_order
will be taken from SHC file. If \emph{k} is greater than or equal to
\begin{quote}

number of temporal snapshots, \emph{k} will be reduced (default \code{-1},
implying set by SHC file).
\end{quote}


\item {} 
\textbf{\texttt{dB}} (\href{https://docs.python.org/library/functions.html\#bool}{\emph{\texttt{bool}}}) -- Return interpolated magnetic field derivative dB/dt instead of
magnetic field (default \code{False}).

\item {} 
\textbf{\texttt{ext}} (\emph{\texttt{\{ 0 \textbar{} 1 \textbar{} 2 \textbar{} 3 \}}}) -- 
If interpolation is performed, determine the behaviour when
extrapolating:
\begin{quote}

0 : return extrapolated value
1 : return 0
2 : raise error (default)
3 : return boundary value
\end{quote}


\end{itemize}

\item[{Returns}] \leavevmode


\item[{Return type}] \leavevmode
numpy.ndarray with shape \code{(N\_times,3,latitude,longtitude)}

\end{description}\end{quote}


\strong{See also:}


{\hyperref[swtools_doc:swtools.get_l_maxmin]{\emph{\code{get\_l\_maxmin()}}}}, \code{read\_gh\_shc()}



\end{fulllineitems}

\index{get\_l\_maxmin() (in module swtools)}

\begin{fulllineitems}
\phantomsection\label{swtools_doc:swtools.get_l_maxmin}\pysiglinewithargsret{\code{swtools.}\bfcode{get\_l\_maxmin}}{\emph{arr\_len}, \emph{lmax=0}, \emph{lmin=0}, \emph{suppress=False}}{}
Semi-brute force attempt to get a reasonable value of lmax
(maximum degree) based on array length

Idea based on the fact that the array length will never exceed
\code{lmax**2}, but \code{lmax} will never be larger than
\code{array length/2}. The algorithm then favours solutions with lower
\code{lmax} where the array length does not correspond to a unique
\code{(lmax,lmin)} pair. \code{lmax} and/or \code{lmin} may be set. If no pair
is found, an error is raised.{}`{}`suppress=True{}`{}` suppresses logger
output.

\end{fulllineitems}

\index{read\_shc() (in module swtools)}

\begin{fulllineitems}
\phantomsection\label{swtools_doc:swtools.read_shc}\pysiglinewithargsret{\code{swtools.}\bfcode{read\_shc}}{\emph{shc\_fn}, \emph{cols='all'}}{}
Read values of gaussian coefficients (g,h) from column(s) in file.

File should be ascii file obeying the SHC format.
\begin{quote}\begin{description}
\item[{Parameters}] \leavevmode\begin{itemize}
\item {} 
\textbf{\texttt{shc\_fn}} (\href{https://docs.python.org/library/functions.html\#str}{\emph{\texttt{str}}}) -- Path of input SHC ascii file

\item {} 
\textbf{\texttt{cols}} (\emph{\texttt{list\_like}}) -- List of columns to read from file. This should correspond to the
columns the different times values coefficients will be
read from. In a standard SHC file the first two columns (0 and 1)
correspond to the degree (l) and order (m) of the harmonic and
should not be included in \emph{cols}. As such the default value
\code{cols='all'} corresponds to \code{cols=range(2,2+N\_times)}, where
\code{N\_times} is the number of time snapshots in the file.

\end{itemize}

\item[{Returns}] \leavevmode

Tuple with following values at given indices:
\begin{enumerate}
\setcounter{enumi}{-1}
\item {} 
numpy.ndarray of gaussian coefficients with such that

\end{enumerate}
\begin{quote}

\code{myarray{[}0{]}} gives all coefficients at the first time point,
given that there are multiple time snapshots. Otherwise
\code{array{[}0{]}} will only contain the first coefficient.
\end{quote}
\begin{enumerate}
\item {} 
spline order \emph{k} as an integer used to reconstruct model from

\end{enumerate}
\begin{quote}

time snapshots.
\end{quote}
\begin{enumerate}
\setcounter{enumi}{1}
\item {} 
number of columns as an integer.

\item {} \begin{description}
\item[{time of the temporal snapshots (in fractional years in the}] \leavevmode
standard SHC format) as 1D \emph{numpy.ndarray}.

\end{description}

\end{enumerate}


\item[{Return type}] \leavevmode
Tuple

\end{description}\end{quote}
\paragraph{Notes}

Missing data values marked as NaN are currently not handled.

\end{fulllineitems}

\index{plot\_align() (in module swtools)}

\begin{fulllineitems}
\phantomsection\label{swtools_doc:swtools.plot_align}\pysiglinewithargsret{\code{swtools.}\bfcode{plot\_align}}{\emph{p1, p2, t1, t2, k=3, align\_to=False, show=False, fmt\_t=True, figsize={[}8.0, 6.0{]}, logx=False, logy=False, legends={[}{]}, lloc='best', lhide=False, colors={[}{]}, **plotkwargs}}{}
Convenience function which combines {\hyperref[swtools_doc:align\string-param]{\emph{align\_param}}} with {\hyperref[swtools_doc:plot\string-basic]{\emph{plot\_basic}}}

Align p1 and p2 using interpolation such that values will be sampled
on the same time steps. Output will be the same as for {\hyperref[swtools_doc:plot\string-basic]{\emph{plot\_basic}}}.

See {\hyperref[swtools_doc:align\string-param]{\emph{align\_param}}} and {\hyperref[swtools_doc:plot\string-basic]{\emph{plot\_basic}}} for more information on arguments.

\end{fulllineitems}

\index{plot\_basic() (in module swtools)}

\begin{fulllineitems}
\phantomsection\label{swtools_doc:swtools.plot_basic}\pysiglinewithargsret{\code{swtools.}\bfcode{plot\_basic}}{\emph{x, y, *xy, show=False, fmt\_t=True, figsize={[}8.0, 6.0{]}, logx=False, logy=False, legends={[}{]}, lloc='best', lhide=False, lbox=False, lfontsize=15, colors={[}{]}, **plotkwargs}}{}~\phantomsection\label{swtools_doc:plot-basic}\begin{quote}

Basic plot using \emph{matplotlib}.
\end{quote}

A convenience function to use \href{http://matplotlib.org/api/pyplot\_api.html\#matplotlib.pyplot.plot}{matplotlib.pyplot.plot}
with some set parameters. Of particular note this function handles an
x-axis with datetimes better than the default behaviour in matplotlib.
\begin{quote}\begin{description}
\item[{Parameters}] \leavevmode\begin{itemize}
\item {} 
\textbf{\texttt{x}} (\emph{\texttt{array\_like}}) -- Input x-values.

\item {} 
\textbf{\texttt{y}} (\emph{\texttt{array\_like}}) -- Input y-values.

\item {} 
\textbf{\texttt{xy}} (\emph{\texttt{optional}}) -- Additional x- and y-values.

\item {} 
\textbf{\texttt{show}} (\emph{\texttt{bool, optional}}) -- Show plot (default \code{False}).

\item {} 
\textbf{\texttt{fmt\_t}} (\emph{\texttt{bool, optional}}) -- Format datetime x-ticks (see \href{http://matplotlib.org/api/figure\_api.html?highlight=autofmt\_xdate}{matplotlib.figure.autofmt\_xdate})
(default \code{True}).

\item {} 
\textbf{\texttt{figsize}} (\emph{\texttt{tuple of length 2, optional}}) -- Size of figure as tuple of width and height in inches
(default \code{matplotlib.pyplot.rcParams{[}"figure.figsize"{]}}).

\item {} 
\textbf{\texttt{logx}} (\emph{\texttt{bool, optional}}) -- Set x-axis scale to log (default \code{False}).

\item {} 
\textbf{\texttt{logy}} (\emph{\texttt{bool, optional}}) -- Set y-axis scale to log (default \code{False}).

\item {} 
\textbf{\texttt{legends}} (\emph{\texttt{list\_like, optional}}) -- Add legend(s)(default \code{{[}{]}}).

\item {} 
\textbf{\texttt{lloc}} (\emph{\texttt{str or int, optional}}) -- Location of legend. Can be one of:
`best' : 0, (default)
`upper right'  : 1,
`upper left'   : 2,
`lower left'   : 3,
`lower right'  : 4,
`right'        : 5,
`center left'  : 6,
`center right' : 7,
`lower center' : 8,
`upper center' : 9,
`center'       : 10

\item {} 
\textbf{\texttt{lhide}} (\emph{\texttt{bool, optional}}) -- Do not show legends. Useful to combine legends with twinx legends
(default \code{False}).

\item {} 
\textbf{\texttt{lbox}} (\href{https://docs.python.org/library/functions.html\#bool}{\emph{\texttt{bool}}}) -- box legends in semi-transparent box (default \code{False})

\item {} 
\textbf{\texttt{lfontsize}} (\emph{\texttt{scalar}}) -- fontsize of legend (default \code{15})

\item {} 
\textbf{\texttt{colors}} (\emph{\texttt{list\_like, optional}}) -- \begin{description}
\item[{Color cycle to use in plot (see {\color{red}\bfseries{}{}`}matplotlib.colors}] \leavevmode
\textless{}\href{http://matplotlib.org/api/colors\_api.html}{http://matplotlib.org/api/colors\_api.html}\textgreater{}{}`\_). Default will use
colormap set in the rcParams

\end{description}

(default \code{{[}{]}}).


\item {} 
\textbf{\texttt{plotkwargs}} (\emph{\texttt{optional}}) -- Additional keyword arguments to pass on to
\href{http://matplotlib.org/api/pyplot\_api.html\#matplotlib.pyplot.plot}{matplotlib.pyplot.plot}, these will be overwritten if conflicting
with other values.

\end{itemize}

\item[{Returns}] \leavevmode
\href{http://matplotlib.org/api/figure\_api.html\#matplotlib.figure.Figure}{matplotlib.figure.Figure}
and
\href{http://matplotlib.org/api/axes\_api.html\#matplotlib.axes.Axes}{matplotlib.axes.Axes}
instances for plot

\item[{Return type}] \leavevmode
\href{https://docs.python.org/library/functions.html\#tuple}{tuple}

\end{description}\end{quote}


\strong{See also:}


{\hyperref[swtools_doc:swtools.plot_twinx]{\emph{\code{plot\_twinx()}}}}, {\hyperref[swtools_doc:swtools.plot_align]{\emph{\code{plot\_align()}}}}



\end{fulllineitems}

\index{plot\_geo() (in module swtools)}

\begin{fulllineitems}
\phantomsection\label{swtools_doc:swtools.plot_geo}\pysiglinewithargsret{\code{swtools.}\bfcode{plot\_geo}}{\emph{lon, lat, param, ptype='scatter', figsize={[}8.0, 6.0{]}, cmap='jet', cbar=True, dark\_map=False, show=False, contourlevels=15, log\_contour=False, show\_lat=True, show\_lon=False, **kwargs}}{}
Plot parameter on the globe using \emph{mpl\_toolkits.basemap.Basemap}.
\begin{quote}\begin{description}
\item[{Parameters}] \leavevmode\begin{itemize}
\item {} 
\textbf{\texttt{lon}} (\emph{\texttt{array\_like}}) -- Longitude of \emph{param}.

\item {} 
\textbf{\texttt{lat}} (\emph{\texttt{array\_like}}) -- Latitude of \emph{param}.

\item {} 
\textbf{\texttt{param}} (\emph{\texttt{array\_like}}) -- Value of \emph{param} at each \code{(lat,lon)}-coordinate.

\item {} 
\textbf{\texttt{ptype}} (\code{\{'scatter'\textbar{}'colormesh'\textbar{}'contour'\}}, optional) -- Set plot type (default \code{'scatter'}).

\item {} 
\textbf{\texttt{figsize}} (\emph{\texttt{tuple, optional}}) -- Size of figure as tuple of width and height in inches
(default \code{matplotlib.pyplot.rcParams{[}"figure.figsize"{]}}).

\item {} 
\textbf{\texttt{cmap}} (\emph{\texttt{matplotlib.colors.ColorMap}}) -- colormap to be used in plot
(default \code{matplotlib.pyplot.rcParams{[}"image.cmap"{]}}).

\item {} 
\textbf{\texttt{cbar}} (\emph{\texttt{bool, optional}}) -- use colorbar (default \code{True}).

\item {} 
\textbf{\texttt{dark\_map}} (\emph{\texttt{bool, optional}}) -- draw map with darker tones of gray (default \code{False}).

\item {} 
\textbf{\texttt{show}} (\emph{\texttt{bool, optional}}) -- Show plot (default \code{False}).

\item {} 
\textbf{\texttt{contourlevels}} (\emph{\texttt{int, optional}}) -- number of coutour levels to use in coutourplot (default \code{15}).

\item {} 
\textbf{\texttt{log\_contour}} (\emph{\texttt{bool, optional}}) -- plot contour levels using logarithmic distances between lines.

\item {} 
\textbf{\texttt{show\_lat}} (\emph{\texttt{bool, optional}}) -- show labels for latitude (default \code{True}).

\item {} 
\textbf{\texttt{show\_lon}} (\emph{\texttt{bool, optional}}) -- show labels for longitude (default \code{False}).

\end{itemize}

\item[{Returns}] \leavevmode
\href{http://matplotlib.org/api/figure\_api.html\#matplotlib.figure.Figure}{matplotlib.figure.Figure} and \href{http://matplotlib.org/basemap/api/basemap\_api.html\#mpl\_toolkits.basemap.Basemap}{mpl\_toolkits.basemap.Basemap}
object for plot.

\item[{Return type}] \leavevmode
\href{https://docs.python.org/library/functions.html\#tuple}{tuple}

\end{description}\end{quote}
\paragraph{Notes}

See \href{http://matplotlib.org/basemap/api/basemap\_api.html}{http://matplotlib.org/basemap/api/basemap\_api.html} for full set
of possible keyword arguments. In particular the projection can be
set with \code{projection}, which is by default set to `moll'
(Mollweide projection) in this funciton. In addition, depending on
the value of \emph{ptype}, the following values are used as default:

\emph{scatter}:
\begin{quote}

See mpl\_toolkits.basemap.scatter
default values:
\begin{itemize}
\item {} 
linewidths : 0.0

\item {} 
vmin : min(param)

\item {} 
vmax : max(param)

\end{itemize}
\end{quote}

\emph{colormesh}:
\begin{quote}

See mpl\_toolkits.basemap.pcolormesh
Note that as \emph{colormesh requires 2D arrays; providing
{}`{}`latlon=True{}`} allows latitude and longitude to be converted to a
2d mesh properly from two 1D arrays.
default values:
\begin{itemize}
\item {} 
shading : flat

\item {} 
alpha : 0.8

\end{itemize}
\end{quote}

\emph{contour}:
\begin{quote}

See mpl\_toolkits.basemap.pcolormesh.contour
Note that as \emph{colormesh requires 2D arrays; providing
{}`{}`latlon=True{}`} allows latitude and longitude to be converted to a
2d mesh properly from two 1D arrays.
default values :
\begin{itemize}
\item {} 
animated : True

\end{itemize}
\end{quote}

\end{fulllineitems}

\index{plot\_scatter() (in module swtools)}

\begin{fulllineitems}
\phantomsection\label{swtools_doc:swtools.plot_scatter}\pysiglinewithargsret{\code{swtools.}\bfcode{plot\_scatter}}{\emph{x, y, param, show=False, fmt\_t=True, figsize={[}8.0, 6.0{]}, vmax=None, vmin=None, cmap='jet', cbar=True, **scatterkwargs}}{}
Scatterplot with colorbar using matplotlib.pyplot.scatter.
\begin{quote}\begin{description}
\item[{Parameters}] \leavevmode\begin{itemize}
\item {} 
\textbf{\texttt{x}} (\emph{\texttt{array\_like}}) -- x-coordinates of \emph{param}.

\item {} 
\textbf{\texttt{y}} (\emph{\texttt{array\_like}}) -- y-coordinates of \emph{param}.

\item {} 
\textbf{\texttt{param}} (\emph{\texttt{array\_like}}) -- value(determining colour) of \emph{param} at each (x,y)-coordinate.

\item {} 
\textbf{\texttt{show}} (\emph{\texttt{bool, optional}}) -- Show plot (default \code{False}).

\item {} 
\textbf{\texttt{fmt\_t}} (\emph{\texttt{bool, optional}}) -- Format datetime x-ticks
(see \href{http://matplotlib.org/api/figure\_api.html?highlight=autofmt\_xdate}{matplotlib.figure.autofmt\_xdate} ) (default \code{True}).

\item {} 
\textbf{\texttt{figsize}} (\emph{\texttt{tuple of length 2, optional}}) -- Size of figure as tuple of width and height in inches
(default \code{matplotlib.pyplot.rcParams{[}"figure.figsize"{]}}).

\item {} 
\textbf{\texttt{vmax}} (\emph{\texttt{scalar, optional}}) -- vmax sets the upper bound of the colour data. If either \emph{vmin} or
\emph{vmax} are None, the min and max of the color array is used
(default \code{None}).

\item {} 
\textbf{\texttt{vmin}} (\emph{\texttt{scalar, optional}}) -- vmin sets the lower bound of the colour data. If either \emph{vmin} or
\emph{vmax} are None, the min and max of the color array is used
(default \code{None}).

\item {} 
\textbf{\texttt{cmap}} (\emph{\texttt{matplotlib.colors.ColorMap}}) -- colormap to be used in plot
(default \code{matplotlib.pyplot.rcParams{[}"image.cmap"{]}}).

\item {} 
\textbf{\texttt{cbar}} (\emph{\texttt{bool, optional}}) -- use colorbar (default \code{True}).

\end{itemize}

\item[{Keyword Arguments}] \leavevmode\begin{itemize}
\item {} 
\textbf{s} (\emph{scalar or array\_like}) --
(size of points)**2 (default \code{3}).

\item {} 
\textbf{linewidths} (\emph{scalar}) --
(default \code{0.0}).

\item {} 
\textbf{alpha} (\emph{scalar}) --
blending value between 0(transparent) and 1(opaque).

\end{itemize}

\item[{Returns}] \leavevmode
\href{http://matplotlib.org/api/figure\_api.html\#matplotlib.figure.Figure}{matplotlib.figure.Figure} and \href{http://matplotlib.org/api/axes\_api.html\#matplotlib.axes.Axes}{matplotlib.axes.Axes}
instances for plot as a tuple

\item[{Return type}] \leavevmode
\href{https://docs.python.org/library/functions.html\#tuple}{tuple}

\end{description}\end{quote}

\end{fulllineitems}

\index{plot\_twinx() (in module swtools)}

\begin{fulllineitems}
\phantomsection\label{swtools_doc:swtools.plot_twinx}\pysiglinewithargsret{\code{swtools.}\bfcode{plot\_twinx}}{\emph{x}, \emph{y}, \emph{*xy}, \emph{show=False}, \emph{logy=False}, \emph{legends={[}{]}}, \emph{lloc='best'}, \emph{lall=True}, \emph{lbox=False}, \emph{lfontsize=15}, \emph{ax=None}, \emph{colors={[}{]}}, \emph{**plotkwargs}}{}
Overplot with a twin x-axis.

Share same x-axis as another plot, but with separate y-axis values.
Should be used in conjunction with another plot function
(eg. {\hyperref[swtools_doc:plot\string-basic]{\emph{plot\_basic}}}).
\begin{quote}\begin{description}
\item[{Parameters}] \leavevmode\begin{itemize}
\item {} 
\textbf{\texttt{x}} (\emph{\texttt{array\_like}}) -- Input x-values.

\item {} 
\textbf{\texttt{y}} (\emph{\texttt{array\_like}}) -- Input y-values.

\item {} 
\textbf{\texttt{xy}} (\emph{\texttt{optional}}) -- Additional x- and y-values.

\item {} 
\textbf{\texttt{show}} (\emph{\texttt{bool, optional}}) -- Show plot (default False).

\item {} 
\textbf{\texttt{logy}} (\emph{\texttt{bool, optional}}) -- Set y-axis scale to log (default \code{False}).

\item {} 
\textbf{\texttt{legends}} (\emph{\texttt{list\_like, optional}}) -- Add legend(s) (default \code{{[}{]}}).

\item {} 
\textbf{\texttt{lloc}} (\emph{\texttt{str or int, optional}}) -- Location of legend. Can be one of::
`best'         : 0 (default)
`upper right'  : 1
`upper left'   : 2
`lower left'   : 3
`lower right'  : 4
`right'        : 5
`center left'  : 6
`center right' : 7
`lower center' : 8
`upper center' : 9
`center'       : 10

\item {} 
\textbf{\texttt{lall}} (\emph{\texttt{bool, optional}}) -- Combine legends from \emph{ax} with \emph{legends} (default \code{True}).

\item {} 
\textbf{\texttt{lbox}} (\href{https://docs.python.org/library/functions.html\#bool}{\emph{\texttt{bool}}}) -- box legends in semi-transparent box (default \code{False})

\item {} 
\textbf{\texttt{lfontsize}} (\emph{\texttt{scalar}}) -- fontsize of legend (default \code{15})

\item {} 
\textbf{\texttt{ax}} (\href{http://matplotlib.org/api/axes\_api.html\#matplotlib.axes.Axes}{\emph{\texttt{matplotlib.axes.Axes}}}) -- Axes instance of plot to share x-axis with. If \code{{}`ax=None{}`}, get
current Axes instance (default None).

\item {} 
\textbf{\texttt{colors}} (\emph{\texttt{list\_like, optional}}) -- \begin{description}
\item[{Color cycle to use in plot (see {\color{red}\bfseries{}{}`}matplotlib.colors}] \leavevmode
\textless{}\href{http://matplotlib.org/api/colors\_api.html}{http://matplotlib.org/api/colors\_api.html}\textgreater{}{}`\_). Default will use
colormap set in the rcParams

\end{description}

(default \code{{[}{]}}).


\item {} 
\textbf{\texttt{plotkwargs}} (\emph{\texttt{optional}}) -- Additional keyword arguments to pass on to
\href{http://matplotlib.org/api/pyplot\_api.html\#matplotlib.pyplot.plot}{matplotlib.pyplot.plot}, these will be overwritten if conflicting
with other values.

\end{itemize}

\item[{Returns}] \leavevmode


\item[{Return type}] \leavevmode
\href{http://matplotlib.org/api/axes\_api.html\#matplotlib.axes.Axes}{matplotlib.axes.Axes}

\end{description}\end{quote}


\strong{See also:}


{\hyperref[swtools_doc:swtools.plot_basic]{\emph{\code{plot\_basic()}}}}, {\hyperref[swtools_doc:swtools.plot_align]{\emph{\code{plot\_align()}}}}



\end{fulllineitems}



\section{Installation requirements}
\label{install:installation-requirements}\label{install::doc}
\code{swtools} requires:

\begin{Verbatim}[commandchars=\\\{\}]
\PYGZhy{} Python (\PYGZgt{}=3.2)
\PYGZhy{} Numpy (\PYGZgt{}=1.5)
\PYGZhy{} Scipy (\PYGZgt{}=0.14)
\PYGZhy{} matplotlib(\PYGZgt{}=1.5)
\PYGZhy{} basemap (\PYGZgt{}=1.0, from mpl\PYGZus{}toolkits)
\PYGZhy{} spacepy (\PYGZgt{}=0.1.5)
\PYGZhy{} numexpr (\PYGZgt{}=2.4)
\PYGZhy{} ftputil (\PYGZgt{}=3.0)
\PYGZhy{} iminuit (\PYGZgt{}=1.0)
\end{Verbatim}

This \emph{should} be all you need to do to get started with swtools:

Install python, C-compile w/ python headers, which for ubuntu the following should suffice:

\begin{Verbatim}[commandchars=\\\{\}]
apt\PYGZhy{}get install build\PYGZhy{}essential python3\PYGZhy{}dev
\end{Verbatim}

then download \href{http://conda.pydata.org/miniconda.html}{miniconda} and run the bash/exe installer.

install required packages:

\begin{Verbatim}[commandchars=\\\{\}]
conda install numpy scipy matplotlib spacepy basemap numexpr pip ipython \PYGZbs{}
  ipython\PYGZhy{}notebook

pip install ftputil iminuit
\end{Verbatim}

Then everything should be ready to be run. If you want to use \code{swtools}, either type in:

\begin{Verbatim}[commandchars=\\\{\}]
python setup.py install
\end{Verbatim}

in the root folder of swtools, or, manually or add \code{swtools} to your pythonpath in a \code{.bash\_profile} or \code{.bashrc} file eg:

\begin{Verbatim}[commandchars=\\\{\}]
export PYTHONPATH=\PYGZdl{}PYTHONPATH:/path/to/swtools/directory
\end{Verbatim}

to use jupyter/ipython notebook just type \code{ipython notebook} in a terminal (optionally add a file path of a notebook file) and it should start up in a browser.


\section{Test}
\label{test::doc}\label{test:test}\phantomsection\label{test:id1}
To test \code{swtools} using \emph{nose}, simply run \code{nosetests} (or alternatively \code{nose2} ) in the \code{tests}-directory.


\section{Remarks}
\label{remarks:remarks}\label{remarks::doc}
Swarm Level0 data products \emph{can} be read, but tools for this have been stored elsewhere as it is less flexible, and builds mainly upon the work of Stefano Mattia, and uses a different framework. To get this, or if there are other questions related to \emph{swtools}, you can contact Mikael Toresen \href{mailto:mikaelt@esa.int}{mikaelt@esa.int}.

A demo of some of the functionality may be found under the \emph{demo} folder.



\renewcommand{\indexname}{Index}
\printindex
\end{document}
